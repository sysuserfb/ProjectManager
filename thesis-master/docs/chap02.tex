\chapter{知识储备}
\section{依赖包管理}
\label{sec:dependencies_manage}
本应用选择npm(Node Package Manager)来对依赖包进行管理,NPM是随同NodeJS一起安装的包管理工具,能解决NodeJS代码部署上的很多问题,常见的功能有:
允许用户从NPM服务器下载别人编写的第三方包到本地使用。
允许用户从NPM服务器下载并安装别人编写的命令行程序到本地使用。
允许用户将自己编写的包或命令行程序上传到NPM服务器供别人使用。
因为本应用需要使用nodejs部署服务器,所以无需其他的包管理器,直接可以使用npm进行包管理,我们只需要在package.json文件上配置所需的依赖,然后`npm install`即可安装所需的依赖包

\section{框架}
\label{sec:frame}
Angular5

angular作为当前最主流的三大前端框架之一,拥有及其完善的功能,而选择它的很重要的一个原因是,相比于vue和react这种轻量级的框架,它更适用于开发大型的,比较复杂的的企业级应用,它拥有完善的项目搭建脚手架,单元测试工具,这对于工程化,规范化的应用开发是十分有利的。
当然,随着版本的更迭,angular也不断吸收vue,react等优秀框架的特点,不断改进自己,使得开发人员使用起来更加方便和得心应手。
使用方法:npm install -g @angular/cli

\section{后端服务器}
\label{sec:server}
express+nodejs

Node.js 是一个基于Chrome JavaScript 运行时建立的一个平台,简单的说它就是运行在服务端的 JavaScript。Node.js是一个事件驱动I/O服务端JavaScript环境,基于Google的V8引擎,V8引擎执行Javascript的速度非常快,性能非常好。Javascript是一个事件驱动语言,Node利用了这个优点,编写出可扩展性高的服务器。Node采用了一个称为“事件循环(event loop)”的架构,使得编写可扩展性高的服务器变得既容易又安全。提高服务器性能的技巧有多种多样。Node选择了一种既能提高性能,又能减低开发复杂度的架构。这是一个非常重要的特性。并发编程通常很复杂且布满地雷。Node绕过了这些,但仍提供很好的性能。
Node采用一系列“非阻塞”库来支持事件循环的方式。本质上就是为文件系统、数据库之类的资源提供接口。向文件系统发送一个请求时,无需等待硬盘(寻址并检索文件),硬盘准备好的时候非阻塞接口会通知Node。该模型以可扩展的方式简化了对慢资源的访问, 直观,易懂。
Node.js使用Module模块去划分不同的功能,以简化应用的开发。Modules模块有点像C++语言中的类库。每一个Node.js的类库都包含了十分丰富的各类函数,比如http模块就包含了和http功能相关的很多函数,可以帮助开发者很容易地对比如http,tcp/udp等进行操作,还可以很容易的创建http和tcp/udp的服务器。
Express 是一个基于 Node.js 平台的极简、灵活的 web 应用开发框架,它提供一系列强大的特性,帮助你创建各种 Web 和移动设备应用,使用 Express 可以快速地搭建一个完整功能的网站。
Express 框架核心特性:
\begin{itemize}
	\item 可以设置中间件来响应 HTTP 请求。
	\item 定义了路由表用于执行不同的 HTTP 请求动作。
	\item 可以通过向模板传递参数来动态渲染 HTML 页面。
\end{itemize}

\section{数据库}
\label{sec:database}
就目前而言,数据库分为关系型(传统型)数据库和非关系型数据库。当前主流的关系型数据库有Oracle、DB2、Microsoft SQL Server、Microsoft Access、MySQL等。非关系型数据库有 NoSql、Cloudant。
两者各有其优缺点,非关系型数据库简单易部署,成本低,查询速度较快,利于数据分割,储存格式是key,value形式、文档形式、图片形式等等,所以可以存储基础类型以及对象或者是集合等各种格式;
而传统的关系型数据库相比于前者,可以做到更加复杂的数据查询,如多表联合查询等,而且其对事务的支持使得对于安全性能很高的数据访问要求得以实现。
由于关系型数据库的各个表之间存在的较强的联系,因此不利于数据的分散,也就是难以将数据分割存储到不同的服务器中,这会导致在数据量太大的时候对服务器的负荷加大

