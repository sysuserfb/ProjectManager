%%
% 结论
% 结论是毕业论文的总结,是整篇论文的归宿,应精炼、准确、完整。结论应着重阐述自己的创造性成果及其在本研究领域中的意义、作用,还可进一步提出需要讨论的问题和建议。
% modifyer: 黄俊杰(huangjj27, 349373001dc@gmail.com)
% update date: 2017-04-13
%%

\chapter{总结与展望}
\section{工作总结}
本应用使用了基于Angular 5 搭建了一个实现了资源包控制线的一系列功能的web应用,不但包括了前端架构的搭建,也涵盖了后台服务器的支持和数据库的操作,上线部署就能开箱即用。
	为达到较好的用户体验和高效的开发过程,本项目使用了一些比较成熟的依赖包,不管是UI上的ng-zorro或是ORM的Sequelize或是文件上传的multer,可以说是建立在了前人的努力上。
	因此,本应用在用户界面上的体验比较好,而且保证了一系列功能的完整性和良好的交互体验,充分利用了前端近年的最新技术是一个很大的优点。
	但是,对于开发者的我来说,在享受着angular框架构建大型项目上的便利性的同时,也面临着接触完全全新的技术的不适应感,angular是由typescript实现的,有一定的学习门槛,对于不熟悉的技术,
	我经常会遇到各种各样不同的问题,搭建前端时也有,实现后台时也有,这样或那样的问题总会导致我的代码比较凌乱,降低了代码的可维护性,甚至增大了出错的可能性。
\section{研究展望}
除了以上的个人技术问题之外,该项目还可以再继续改进,比如如何优化访问速度,如何优化运行的性能,如何保证数据的安全性,如何安全验证用户登录等等。
因此,本项目仍有很大的发展空间,比如在界面上进一步优化,在代码上进一步精简,在数据库的操作上更加规范合理。当然,项目本身的可扩展性是很强大的,未来可以添加一个数据分析模块,后台管理员模块,模块化编程使得我们不需要改动太多的代码就可以新增一个功能。
\clearpage
