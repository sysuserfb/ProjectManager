%%
% 开题报告
% modifier: 黄俊杰(huangjj27, 349373001dc@gmail.com)
% update date: 2017-05-14

% 选题目的
\objective{
时代在进步,技术在发展,我们需要改变自己的观念,在摸索中不断尝试新的事物。
本论文基于企业的产品线控制的需求,致力于建立一个完善的资源包管理应用,为产品版本的开发,测试,审核,发布等流程的管理提出的一站式解决方案。
技术实现上,本应用基于目前较成熟的angular框架,实现资源包控制线的一系列功能,在实践中发掘目前前端技术上的优缺点,改变前端的思维模式。
}

% 思路
\methodology{
该应用需要包括三方面的功能:

账户管理的功能包括登录,注册,注销,更改个人信息,基本满足了对于应用使用者的需求
产品管理的功能包括修改产品信息,添加某角色(产品开发者/测试员)的成员,删除某角色的成员,更换管理员,新建产品(即提交新产品的信息和新建产品的第一个版本)。
版本管理的功能包括新建版本(需要上传新版本的资源包),版本流程的管理(测试,审核,发布),版本回滚。
最后还有成员管理,这个功能其实包括在产品管理里面,这里主要说明一下成员和用户的区别。
两者之间的区别在于,一个用户可以是多个成员,甚至是多个不同产品的相同或者不同的成员,而每个产品下都有三种成员:产品管理者,产品开发者,产品测试员,他们拥有不同的权限。
		
}

% 研究方法/程序/步骤
\researchProcedure{
\begin{itemize}
	\item 选用合适的前端框架和服务器实现技术
	\item 搜索相关的资料进行学习
	\item 设计交互页面以及所需数据
	\item 设计数据库表格
	\item 编写接口文档
	\item 搭建项目,分模块实现功能
	\item 编写后台,按文档实现接口
	\item 前后端对接,调试改进
\end{itemize}
}

% 相关支持条件
\supportment{
	Windows系统
	
	安装配置nodejs
	
	安装angular
}

% 进度安排
\schedule{
根据上面的步骤进行安排:

1月,搜索资料,设计数据库以及页面

2月,搭建前端项目

3月,实现后台接口

4月,前后端对接,调试改进
}

% 指导老师意见
\proposalInstructions{

}

