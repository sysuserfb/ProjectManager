%%
% 引言或背景
% 引言是论文正文的开端,应包括毕业论文选题的背景、目的和意义;对国内外研究现状和相关领域中已有的研究成果的简要评述;介绍本项研究工作研究设想、研究方法或实验设计、理论依据或实验基础;涉及范围和预期结果等。要求言简意赅,注意不要与摘要雷同或成为摘要的注解。
% modifier: 黄俊杰(huangjj27, 349373001dc@gmail.com)
% update date: 2017-04-15
%%

\chapter{引言}
%定义,过去的研究和现在的研究,意义,与图像分割的不同,going deeper
\label{cha:introduction}
\section{选题背景与意义}
\label{sec:background}
% What is the problem
% why is it interesting and important
% Why is it hards, why do naive approaches fails
% why hasn't it been solved before
% what are the key components of my approach and results, also include any specific limitations,do not repeat the abstract
%contribution
Web 发展了几十个春秋,风起云涌,千变万化。Web 技术发展的速度让人感觉那几乎不是继承式的迭代,而是一次又一次的变革,一次又一次的创造。
所谓前端就是html+css吗,可能以前是,但现在已经不再是这样了。随着前端技术的发展以及观念的变化,引领技术潮流的几大巨头也在不断地创新和发展,不断提出新的概念,企图超越别人,扩大影响力。
到2009 年之后,JavaScript 类库已经颇为成熟,各大类库也是相互吸收优点,不断完善并提高自身性能,然而功能上已经没有太多增加的势头。部分框架开始了思想上的转变,更加注重前端开发的组织和结构,条理性强了很多,如 YUI 等。
ECMAScript 规范的争执,开启了浏览器引擎大战,各大厂商也趁机瓜分 IE 份额,Chrome 和 Firefox 在这场战役中取得大胜。jQuery 火了,成千上万的 JQ 插件让网页开发变得尤为轻松,而随之而来的也是页面的臃肿和性能调优的深入探索。
因此,前端风向渐渐转向工程化和模块化,前端工程化开始普及,各公司开始推出自己的前端集成开发解决方案。Node.js 前后端分离的流行,中间层的出现改变了前后端的合作模式。
纵观前端技术发展,只从前端应用开发框架上来看,就先后经历了DOM API、MVC、MVP、MVVM、Virtual DOM、MNV*阶段,逐步解决了前端开发效率、设计模式、DOM交互性能的问题。
时代在进步,技术在发展,我们需要改变自己的观念,在摸索中不断尝试新的事物。本论文基于企业的产品线控制的需求,致力于建立一个完善的资源包管理的一站式解决方案。技术实现上,本应用基于目前较成熟的angular框架,实现资源包控制线的一系列功能,在实践中发掘目前前端技术上的优缺点,改变前端的思维模式。

\section{国内外研究现状和相关工作}
\label{sec:related_work}
在规范上,前端新标准和草案在不断更新,HTML、CSS、JavaScript标准也在渐渐完善,经过大版本的更新稳定,目前前端三层结构实现已经形成了HTML5、CSS3、EcmaScript 6+标准规范结合的阶段。
2011 年 HTML5 的技术发展和推广都向前迈进了一大步,语义明确的标签体系、简洁明了的富媒体支持、本地数据的储存技术、canvas 等等各类技术被广泛应用。相反,flash技术开始逐渐凋零。

在思维模式上,前端开发逐渐趋向工程化和模块化,模块化是一种处理复杂系统分解为代码结构更合理,可维护性更高的可管理的模块的方式。理想状态下我们只需要完成自己部分的核心业务逻辑代码,其他方面的依赖可以通过直接加载被人已经写好模块进行使用即可。在前端工程化上,几个派系相互争斗,产出了 AMD、CMD、UMD 等规范,也衍生了 SeaJS、RequireJS 等模块化工具。

在框架上,angular,vue,react作为当前前端市场的三大主流框架,被绝大多数的开发人员所使用着。
angular是Google支持并开发维护的JavaScript框架,从2009年AngularJS诞生开始,Google公司不断对这个框架于进行改进和升级,2017年 11月1日 Google公司发布了 Angular 5,该版本可支持高新能、离线使用、面安装的app式体验,并借鉴了来自Ionic、NativeScript和React Native中的技术与思想,构建原生移动应用,能够横跨Mac,Windows,Linux平台,并可以自动生成和拆分代码,提高生产效率。
除此之外,angular还有简单强大了模板语法,快速创建UI视图,还有强大的项目构建工具Angular-CLI和测试工具Karma
angular是一个比较完善的前端MVW(Model-View-Whatever)框架,包含模板,数据双向绑定,路由,模块化,服务,依赖注入等所有功能,模板功能强大丰富,并且是声明式的,自带了丰富的 Angular 指令,AngularJS 通过 指令 扩展了 HTML,且通过 表达式 绑定数据到 HTML,而且它是以一个 JavaScript 文件形式发布的,可通过 script 标签添加到网页中。

\section{本文的论文结构与章节安排}
\label{sec:arrangement}
本文共分为五章,各章节内容安排如下:

第一章引言。

第二章知识储备。

第三章方法介绍。

第四章实验和结果。

第五章是本文的最后一章,总结与展望。是对本文内容的整体性总结以及对未来工作的展望。

